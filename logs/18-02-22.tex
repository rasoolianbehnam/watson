\documentclass{article}
%easy margins
\usepackage[top=.2in, bottom=.4in, left=.8in, right=.8in]{geometry}
%http links in text
\usepackage{hyperref}
%pictures
\usepackage{graphicx}
%quotes
\usepackage{csquotes}
%landscape single pages
\usepackage{lscape}
%inline enumeration
\usepackage[inline]{enumitem}
%to change spacing
\usepackage{setspace}

\usepackage{float}
%****************************************************************
% changing font family
\renewcommand{\familydefault}{\sfdefault} %change font

%%%%%%%%%%%%%%%%%%%%%%%%%%%%%%%%%%%%%%%%%%%%%%%%%%%%%%%%%%%%%%%%%%%
%Adding title and authors
%\title{Cell classification by applying graphlets on HiC data}
%\author{Behnam Rasoolian  \and Liangliang Xu \and Zheng Zhang}
%\date{}
%%%%%%%%%%%%%%%%%%%%%%%%%%%%%%%%%%%%%%%%%%%%%%%%%%%%%%%%%%%%%%%%%%%

%%%%%%%%%%%%%%%%%%%%%%%%%%%%%%%%%%%%%%%%%%%%%%%%%%%%%%%%%%%%%%%%%%%
%Chaning font to a local font
%\setmainfont[
%SmallCapsFont = Fontin-SmallCaps.otf,
%BoldFont = Fontin-Bold.otf,
%ItalicFont = Fontin-Italic.otf
%]
%{Fontin.otf}
%%%%%%%%%%%%%%%%%%%%%%%%%%%%%%%%%%%%%%%%%%%%%%%%%%%%%%%%%%%%%%%%%%%
\begin{document}
%%%%%%%%%%%%%%%%%%%%%%%%%%%%%%%%%%%%%%%%%%%%%%%%%%%%%%%%%%%%%%%%%%%
%%page title
%\maketitle
%%%%%%%%%%%%%%%%%%%%%%%%%%%%%%%%%%%%%%%%%%%%%%%%%%%%%%%%%%%%%%%%%%%
%\doublespacing
% ****setstrech should always go before fontsize
%\setstretch{2.0}
%%%%%%%%%%%%%%%%%%%%%%%%%%%%%%%%%%%%%%%%%%%%%%%%%%%%%%%%%%%%%%%%%%%
%%micromanage font size, the first argument is
%%size of the font, the second line distance
\fontsize{13pt}{26pt}\selectfont

%No page numbers.
\pagenumbering{gobble}
%no indentaion for new paragraphs.
\setlength{\parindent}{0pt}
I have extracted graphlets for all 23 chromosomes from
the contact maps, using ice normalization and I
have thresholed it (test.py). I then separated individual
orbitals from each cell category and put them together.
That is, I have put orbital 0 from MIT, ALL, RL and CALL4
together and put them in a file. Then I did the same for orbital
1 and put them in a separate file and so forth. I then ran MINE.jar
for each file and captured pairwise MIC values for each orbital.
By doing so, the result would have a dataset of the follwoing shape:
\begin{table}[H]
    \centering
    \begin{tabular}{|c|c|c|c|}
        \hline
        Cell 1 & Cell 2 & Orbital & MIC value\\ \hline
    \end{tabular}
\end{table}
My plan was to find a signinficance of pair-wice MIC values 
fo example, using a
null hypothesis of $H_0: MIC_{MIT-ALL} = 0$. This would mean
That MIT-ALL are totally independent of each other. However,
What we are trying to prove is that they are not the same,
which is different. Actually I have to test this null hypothesis:\\
$H_0: MIC_{MIT-ALL} = 1$\\
which if cannot be rejected, means that the two sets of MIT and ALL
are similar to each other.

In order to validate my method, I have to be able to compare the two
normal cell data that I have, namely, MIT and the one from Rao et. al.
I have to be able to get a high value between the two.

\end{document}
