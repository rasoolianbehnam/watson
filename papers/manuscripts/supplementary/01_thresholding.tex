
% Default to the notebook output style

    


% Inherit from the specified cell style.




    
\documentclass[11pt]{article}

    
    
    \usepackage[T1]{fontenc}
    % Nicer default font (+ math font) than Computer Modern for most use cases
    \usepackage{mathpazo}

    % Basic figure setup, for now with no caption control since it's done
    % automatically by Pandoc (which extracts ![](path) syntax from Markdown).
    \usepackage{graphicx}
    % We will generate all images so they have a width \maxwidth. This means
    % that they will get their normal width if they fit onto the page, but
    % are scaled down if they would overflow the margins.
    \makeatletter
    \def\maxwidth{\ifdim\Gin@nat@width>\linewidth\linewidth
    \else\Gin@nat@width\fi}
    \makeatother
    \let\Oldincludegraphics\includegraphics
    % Set max figure width to be 80% of text width, for now hardcoded.
    \renewcommand{\includegraphics}[1]{\Oldincludegraphics[width=.8\maxwidth]{#1}}
    % Ensure that by default, figures have no caption (until we provide a
    % proper Figure object with a Caption API and a way to capture that
    % in the conversion process - todo).
    \usepackage{caption}
    \DeclareCaptionLabelFormat{nolabel}{}
    \captionsetup{labelformat=nolabel}

    \usepackage{adjustbox} % Used to constrain images to a maximum size 
    \usepackage{xcolor} % Allow colors to be defined
    \usepackage{enumerate} % Needed for markdown enumerations to work
    \usepackage{geometry} % Used to adjust the document margins
    \usepackage{amsmath} % Equations
    \usepackage{amssymb} % Equations
    \usepackage{textcomp} % defines textquotesingle
    % Hack from http://tex.stackexchange.com/a/47451/13684:
    \AtBeginDocument{%
        \def\PYZsq{\textquotesingle}% Upright quotes in Pygmentized code
    }
    \usepackage{upquote} % Upright quotes for verbatim code
    \usepackage{eurosym} % defines \euro
    \usepackage[mathletters]{ucs} % Extended unicode (utf-8) support
    \usepackage[utf8x]{inputenc} % Allow utf-8 characters in the tex document
    \usepackage{fancyvrb} % verbatim replacement that allows latex
    \usepackage{grffile} % extends the file name processing of package graphics 
                         % to support a larger range 
    % The hyperref package gives us a pdf with properly built
    % internal navigation ('pdf bookmarks' for the table of contents,
    % internal cross-reference links, web links for URLs, etc.)
    \usepackage{hyperref}
    \usepackage{longtable} % longtable support required by pandoc >1.10
    \usepackage{booktabs}  % table support for pandoc > 1.12.2
    \usepackage[inline]{enumitem} % IRkernel/repr support (it uses the enumerate* environment)
    \usepackage[normalem]{ulem} % ulem is needed to support strikethroughs (\sout)
                                % normalem makes italics be italics, not underlines
    

    
    
    % Colors for the hyperref package
    \definecolor{urlcolor}{rgb}{0,.145,.698}
    \definecolor{linkcolor}{rgb}{.71,0.21,0.01}
    \definecolor{citecolor}{rgb}{.12,.54,.11}

    % ANSI colors
    \definecolor{ansi-black}{HTML}{3E424D}
    \definecolor{ansi-black-intense}{HTML}{282C36}
    \definecolor{ansi-red}{HTML}{E75C58}
    \definecolor{ansi-red-intense}{HTML}{B22B31}
    \definecolor{ansi-green}{HTML}{00A250}
    \definecolor{ansi-green-intense}{HTML}{007427}
    \definecolor{ansi-yellow}{HTML}{DDB62B}
    \definecolor{ansi-yellow-intense}{HTML}{B27D12}
    \definecolor{ansi-blue}{HTML}{208FFB}
    \definecolor{ansi-blue-intense}{HTML}{0065CA}
    \definecolor{ansi-magenta}{HTML}{D160C4}
    \definecolor{ansi-magenta-intense}{HTML}{A03196}
    \definecolor{ansi-cyan}{HTML}{60C6C8}
    \definecolor{ansi-cyan-intense}{HTML}{258F8F}
    \definecolor{ansi-white}{HTML}{C5C1B4}
    \definecolor{ansi-white-intense}{HTML}{A1A6B2}

    % commands and environments needed by pandoc snippets
    % extracted from the output of `pandoc -s`
    \providecommand{\tightlist}{%
      \setlength{\itemsep}{0pt}\setlength{\parskip}{0pt}}
    \DefineVerbatimEnvironment{Highlighting}{Verbatim}{commandchars=\\\{\}}
    % Add ',fontsize=\small' for more characters per line
    \newenvironment{Shaded}{}{}
    \newcommand{\KeywordTok}[1]{\textcolor[rgb]{0.00,0.44,0.13}{\textbf{{#1}}}}
    \newcommand{\DataTypeTok}[1]{\textcolor[rgb]{0.56,0.13,0.00}{{#1}}}
    \newcommand{\DecValTok}[1]{\textcolor[rgb]{0.25,0.63,0.44}{{#1}}}
    \newcommand{\BaseNTok}[1]{\textcolor[rgb]{0.25,0.63,0.44}{{#1}}}
    \newcommand{\FloatTok}[1]{\textcolor[rgb]{0.25,0.63,0.44}{{#1}}}
    \newcommand{\CharTok}[1]{\textcolor[rgb]{0.25,0.44,0.63}{{#1}}}
    \newcommand{\StringTok}[1]{\textcolor[rgb]{0.25,0.44,0.63}{{#1}}}
    \newcommand{\CommentTok}[1]{\textcolor[rgb]{0.38,0.63,0.69}{\textit{{#1}}}}
    \newcommand{\OtherTok}[1]{\textcolor[rgb]{0.00,0.44,0.13}{{#1}}}
    \newcommand{\AlertTok}[1]{\textcolor[rgb]{1.00,0.00,0.00}{\textbf{{#1}}}}
    \newcommand{\FunctionTok}[1]{\textcolor[rgb]{0.02,0.16,0.49}{{#1}}}
    \newcommand{\RegionMarkerTok}[1]{{#1}}
    \newcommand{\ErrorTok}[1]{\textcolor[rgb]{1.00,0.00,0.00}{\textbf{{#1}}}}
    \newcommand{\NormalTok}[1]{{#1}}
    
    % Additional commands for more recent versions of Pandoc
    \newcommand{\ConstantTok}[1]{\textcolor[rgb]{0.53,0.00,0.00}{{#1}}}
    \newcommand{\SpecialCharTok}[1]{\textcolor[rgb]{0.25,0.44,0.63}{{#1}}}
    \newcommand{\VerbatimStringTok}[1]{\textcolor[rgb]{0.25,0.44,0.63}{{#1}}}
    \newcommand{\SpecialStringTok}[1]{\textcolor[rgb]{0.73,0.40,0.53}{{#1}}}
    \newcommand{\ImportTok}[1]{{#1}}
    \newcommand{\DocumentationTok}[1]{\textcolor[rgb]{0.73,0.13,0.13}{\textit{{#1}}}}
    \newcommand{\AnnotationTok}[1]{\textcolor[rgb]{0.38,0.63,0.69}{\textbf{\textit{{#1}}}}}
    \newcommand{\CommentVarTok}[1]{\textcolor[rgb]{0.38,0.63,0.69}{\textbf{\textit{{#1}}}}}
    \newcommand{\VariableTok}[1]{\textcolor[rgb]{0.10,0.09,0.49}{{#1}}}
    \newcommand{\ControlFlowTok}[1]{\textcolor[rgb]{0.00,0.44,0.13}{\textbf{{#1}}}}
    \newcommand{\OperatorTok}[1]{\textcolor[rgb]{0.40,0.40,0.40}{{#1}}}
    \newcommand{\BuiltInTok}[1]{{#1}}
    \newcommand{\ExtensionTok}[1]{{#1}}
    \newcommand{\PreprocessorTok}[1]{\textcolor[rgb]{0.74,0.48,0.00}{{#1}}}
    \newcommand{\AttributeTok}[1]{\textcolor[rgb]{0.49,0.56,0.16}{{#1}}}
    \newcommand{\InformationTok}[1]{\textcolor[rgb]{0.38,0.63,0.69}{\textbf{\textit{{#1}}}}}
    \newcommand{\WarningTok}[1]{\textcolor[rgb]{0.38,0.63,0.69}{\textbf{\textit{{#1}}}}}
    
    
    % Define a nice break command that doesn't care if a line doesn't already
    % exist.
    \def\br{\hspace*{\fill} \\* }
    % Math Jax compatability definitions
    \def\gt{>}
    \def\lt{<}
    % Document parameters
    \title{01\_thresholding}
    
    
    

    % Pygments definitions
    
\makeatletter
\def\PY@reset{\let\PY@it=\relax \let\PY@bf=\relax%
    \let\PY@ul=\relax \let\PY@tc=\relax%
    \let\PY@bc=\relax \let\PY@ff=\relax}
\def\PY@tok#1{\csname PY@tok@#1\endcsname}
\def\PY@toks#1+{\ifx\relax#1\empty\else%
    \PY@tok{#1}\expandafter\PY@toks\fi}
\def\PY@do#1{\PY@bc{\PY@tc{\PY@ul{%
    \PY@it{\PY@bf{\PY@ff{#1}}}}}}}
\def\PY#1#2{\PY@reset\PY@toks#1+\relax+\PY@do{#2}}

\expandafter\def\csname PY@tok@gd\endcsname{\def\PY@tc##1{\textcolor[rgb]{0.63,0.00,0.00}{##1}}}
\expandafter\def\csname PY@tok@gu\endcsname{\let\PY@bf=\textbf\def\PY@tc##1{\textcolor[rgb]{0.50,0.00,0.50}{##1}}}
\expandafter\def\csname PY@tok@gt\endcsname{\def\PY@tc##1{\textcolor[rgb]{0.00,0.27,0.87}{##1}}}
\expandafter\def\csname PY@tok@gs\endcsname{\let\PY@bf=\textbf}
\expandafter\def\csname PY@tok@gr\endcsname{\def\PY@tc##1{\textcolor[rgb]{1.00,0.00,0.00}{##1}}}
\expandafter\def\csname PY@tok@cm\endcsname{\let\PY@it=\textit\def\PY@tc##1{\textcolor[rgb]{0.25,0.50,0.50}{##1}}}
\expandafter\def\csname PY@tok@vg\endcsname{\def\PY@tc##1{\textcolor[rgb]{0.10,0.09,0.49}{##1}}}
\expandafter\def\csname PY@tok@vi\endcsname{\def\PY@tc##1{\textcolor[rgb]{0.10,0.09,0.49}{##1}}}
\expandafter\def\csname PY@tok@vm\endcsname{\def\PY@tc##1{\textcolor[rgb]{0.10,0.09,0.49}{##1}}}
\expandafter\def\csname PY@tok@mh\endcsname{\def\PY@tc##1{\textcolor[rgb]{0.40,0.40,0.40}{##1}}}
\expandafter\def\csname PY@tok@cs\endcsname{\let\PY@it=\textit\def\PY@tc##1{\textcolor[rgb]{0.25,0.50,0.50}{##1}}}
\expandafter\def\csname PY@tok@ge\endcsname{\let\PY@it=\textit}
\expandafter\def\csname PY@tok@vc\endcsname{\def\PY@tc##1{\textcolor[rgb]{0.10,0.09,0.49}{##1}}}
\expandafter\def\csname PY@tok@il\endcsname{\def\PY@tc##1{\textcolor[rgb]{0.40,0.40,0.40}{##1}}}
\expandafter\def\csname PY@tok@go\endcsname{\def\PY@tc##1{\textcolor[rgb]{0.53,0.53,0.53}{##1}}}
\expandafter\def\csname PY@tok@cp\endcsname{\def\PY@tc##1{\textcolor[rgb]{0.74,0.48,0.00}{##1}}}
\expandafter\def\csname PY@tok@gi\endcsname{\def\PY@tc##1{\textcolor[rgb]{0.00,0.63,0.00}{##1}}}
\expandafter\def\csname PY@tok@gh\endcsname{\let\PY@bf=\textbf\def\PY@tc##1{\textcolor[rgb]{0.00,0.00,0.50}{##1}}}
\expandafter\def\csname PY@tok@ni\endcsname{\let\PY@bf=\textbf\def\PY@tc##1{\textcolor[rgb]{0.60,0.60,0.60}{##1}}}
\expandafter\def\csname PY@tok@nl\endcsname{\def\PY@tc##1{\textcolor[rgb]{0.63,0.63,0.00}{##1}}}
\expandafter\def\csname PY@tok@nn\endcsname{\let\PY@bf=\textbf\def\PY@tc##1{\textcolor[rgb]{0.00,0.00,1.00}{##1}}}
\expandafter\def\csname PY@tok@no\endcsname{\def\PY@tc##1{\textcolor[rgb]{0.53,0.00,0.00}{##1}}}
\expandafter\def\csname PY@tok@na\endcsname{\def\PY@tc##1{\textcolor[rgb]{0.49,0.56,0.16}{##1}}}
\expandafter\def\csname PY@tok@nb\endcsname{\def\PY@tc##1{\textcolor[rgb]{0.00,0.50,0.00}{##1}}}
\expandafter\def\csname PY@tok@nc\endcsname{\let\PY@bf=\textbf\def\PY@tc##1{\textcolor[rgb]{0.00,0.00,1.00}{##1}}}
\expandafter\def\csname PY@tok@nd\endcsname{\def\PY@tc##1{\textcolor[rgb]{0.67,0.13,1.00}{##1}}}
\expandafter\def\csname PY@tok@ne\endcsname{\let\PY@bf=\textbf\def\PY@tc##1{\textcolor[rgb]{0.82,0.25,0.23}{##1}}}
\expandafter\def\csname PY@tok@nf\endcsname{\def\PY@tc##1{\textcolor[rgb]{0.00,0.00,1.00}{##1}}}
\expandafter\def\csname PY@tok@si\endcsname{\let\PY@bf=\textbf\def\PY@tc##1{\textcolor[rgb]{0.73,0.40,0.53}{##1}}}
\expandafter\def\csname PY@tok@s2\endcsname{\def\PY@tc##1{\textcolor[rgb]{0.73,0.13,0.13}{##1}}}
\expandafter\def\csname PY@tok@nt\endcsname{\let\PY@bf=\textbf\def\PY@tc##1{\textcolor[rgb]{0.00,0.50,0.00}{##1}}}
\expandafter\def\csname PY@tok@nv\endcsname{\def\PY@tc##1{\textcolor[rgb]{0.10,0.09,0.49}{##1}}}
\expandafter\def\csname PY@tok@s1\endcsname{\def\PY@tc##1{\textcolor[rgb]{0.73,0.13,0.13}{##1}}}
\expandafter\def\csname PY@tok@dl\endcsname{\def\PY@tc##1{\textcolor[rgb]{0.73,0.13,0.13}{##1}}}
\expandafter\def\csname PY@tok@ch\endcsname{\let\PY@it=\textit\def\PY@tc##1{\textcolor[rgb]{0.25,0.50,0.50}{##1}}}
\expandafter\def\csname PY@tok@m\endcsname{\def\PY@tc##1{\textcolor[rgb]{0.40,0.40,0.40}{##1}}}
\expandafter\def\csname PY@tok@gp\endcsname{\let\PY@bf=\textbf\def\PY@tc##1{\textcolor[rgb]{0.00,0.00,0.50}{##1}}}
\expandafter\def\csname PY@tok@sh\endcsname{\def\PY@tc##1{\textcolor[rgb]{0.73,0.13,0.13}{##1}}}
\expandafter\def\csname PY@tok@ow\endcsname{\let\PY@bf=\textbf\def\PY@tc##1{\textcolor[rgb]{0.67,0.13,1.00}{##1}}}
\expandafter\def\csname PY@tok@sx\endcsname{\def\PY@tc##1{\textcolor[rgb]{0.00,0.50,0.00}{##1}}}
\expandafter\def\csname PY@tok@bp\endcsname{\def\PY@tc##1{\textcolor[rgb]{0.00,0.50,0.00}{##1}}}
\expandafter\def\csname PY@tok@c1\endcsname{\let\PY@it=\textit\def\PY@tc##1{\textcolor[rgb]{0.25,0.50,0.50}{##1}}}
\expandafter\def\csname PY@tok@fm\endcsname{\def\PY@tc##1{\textcolor[rgb]{0.00,0.00,1.00}{##1}}}
\expandafter\def\csname PY@tok@o\endcsname{\def\PY@tc##1{\textcolor[rgb]{0.40,0.40,0.40}{##1}}}
\expandafter\def\csname PY@tok@kc\endcsname{\let\PY@bf=\textbf\def\PY@tc##1{\textcolor[rgb]{0.00,0.50,0.00}{##1}}}
\expandafter\def\csname PY@tok@c\endcsname{\let\PY@it=\textit\def\PY@tc##1{\textcolor[rgb]{0.25,0.50,0.50}{##1}}}
\expandafter\def\csname PY@tok@mf\endcsname{\def\PY@tc##1{\textcolor[rgb]{0.40,0.40,0.40}{##1}}}
\expandafter\def\csname PY@tok@err\endcsname{\def\PY@bc##1{\setlength{\fboxsep}{0pt}\fcolorbox[rgb]{1.00,0.00,0.00}{1,1,1}{\strut ##1}}}
\expandafter\def\csname PY@tok@mb\endcsname{\def\PY@tc##1{\textcolor[rgb]{0.40,0.40,0.40}{##1}}}
\expandafter\def\csname PY@tok@ss\endcsname{\def\PY@tc##1{\textcolor[rgb]{0.10,0.09,0.49}{##1}}}
\expandafter\def\csname PY@tok@sr\endcsname{\def\PY@tc##1{\textcolor[rgb]{0.73,0.40,0.53}{##1}}}
\expandafter\def\csname PY@tok@mo\endcsname{\def\PY@tc##1{\textcolor[rgb]{0.40,0.40,0.40}{##1}}}
\expandafter\def\csname PY@tok@kd\endcsname{\let\PY@bf=\textbf\def\PY@tc##1{\textcolor[rgb]{0.00,0.50,0.00}{##1}}}
\expandafter\def\csname PY@tok@mi\endcsname{\def\PY@tc##1{\textcolor[rgb]{0.40,0.40,0.40}{##1}}}
\expandafter\def\csname PY@tok@kn\endcsname{\let\PY@bf=\textbf\def\PY@tc##1{\textcolor[rgb]{0.00,0.50,0.00}{##1}}}
\expandafter\def\csname PY@tok@cpf\endcsname{\let\PY@it=\textit\def\PY@tc##1{\textcolor[rgb]{0.25,0.50,0.50}{##1}}}
\expandafter\def\csname PY@tok@kr\endcsname{\let\PY@bf=\textbf\def\PY@tc##1{\textcolor[rgb]{0.00,0.50,0.00}{##1}}}
\expandafter\def\csname PY@tok@s\endcsname{\def\PY@tc##1{\textcolor[rgb]{0.73,0.13,0.13}{##1}}}
\expandafter\def\csname PY@tok@kp\endcsname{\def\PY@tc##1{\textcolor[rgb]{0.00,0.50,0.00}{##1}}}
\expandafter\def\csname PY@tok@w\endcsname{\def\PY@tc##1{\textcolor[rgb]{0.73,0.73,0.73}{##1}}}
\expandafter\def\csname PY@tok@kt\endcsname{\def\PY@tc##1{\textcolor[rgb]{0.69,0.00,0.25}{##1}}}
\expandafter\def\csname PY@tok@sc\endcsname{\def\PY@tc##1{\textcolor[rgb]{0.73,0.13,0.13}{##1}}}
\expandafter\def\csname PY@tok@sb\endcsname{\def\PY@tc##1{\textcolor[rgb]{0.73,0.13,0.13}{##1}}}
\expandafter\def\csname PY@tok@sa\endcsname{\def\PY@tc##1{\textcolor[rgb]{0.73,0.13,0.13}{##1}}}
\expandafter\def\csname PY@tok@k\endcsname{\let\PY@bf=\textbf\def\PY@tc##1{\textcolor[rgb]{0.00,0.50,0.00}{##1}}}
\expandafter\def\csname PY@tok@se\endcsname{\let\PY@bf=\textbf\def\PY@tc##1{\textcolor[rgb]{0.73,0.40,0.13}{##1}}}
\expandafter\def\csname PY@tok@sd\endcsname{\let\PY@it=\textit\def\PY@tc##1{\textcolor[rgb]{0.73,0.13,0.13}{##1}}}

\def\PYZbs{\char`\\}
\def\PYZus{\char`\_}
\def\PYZob{\char`\{}
\def\PYZcb{\char`\}}
\def\PYZca{\char`\^}
\def\PYZam{\char`\&}
\def\PYZlt{\char`\<}
\def\PYZgt{\char`\>}
\def\PYZsh{\char`\#}
\def\PYZpc{\char`\%}
\def\PYZdl{\char`\$}
\def\PYZhy{\char`\-}
\def\PYZsq{\char`\'}
\def\PYZdq{\char`\"}
\def\PYZti{\char`\~}
% for compatibility with earlier versions
\def\PYZat{@}
\def\PYZlb{[}
\def\PYZrb{]}
\makeatother


    % Exact colors from NB
    \definecolor{incolor}{rgb}{0.0, 0.0, 0.5}
    \definecolor{outcolor}{rgb}{0.545, 0.0, 0.0}



    
    % Prevent overflowing lines due to hard-to-break entities
    \sloppy 
    % Setup hyperref package
    \hypersetup{
      breaklinks=true,  % so long urls are correctly broken across lines
      colorlinks=true,
      urlcolor=urlcolor,
      linkcolor=linkcolor,
      citecolor=citecolor,
      }
    % Slightly bigger margins than the latex defaults
    
    \geometry{verbose,tmargin=1in,bmargin=1in,lmargin=1in,rmargin=1in}
    
    

    \begin{document}
    
    
    \maketitle
    
    

    
    \section{Local Thresholding}\label{local-thresholding}

In this notebook, I extract orbits for a signle chromosome in all 4 cell
lines, using a different method of thresholding. In order to do this, I
slide a kernel of size k through each pixel and check if it has a
particular property. If the property is satisified, then the pixel is
set to 1, otherwise it is set to 0. The two properties that I experiment
with are \emph{maximum thresholding} and \emph{normal thresholding}. 1.
\textbf{Maximum Thresholding}: in max thresholding, if the pixel is the
maximum of its neighbors with respect to the kernel, then it is set.

\begin{Shaded}
\begin{Highlighting}[]
\KeywordTok{def}\NormalTok{ local_threshold(mat, k}\OperatorTok{=}\DecValTok{1}\NormalTok{, method}\OperatorTok{=}\StringTok{'max'}\NormalTok{, t}\OperatorTok{=}\DecValTok{1}\NormalTok{):}
\NormalTok{    mat2 }\OperatorTok{=}\NormalTok{ np.zeros_like(mat)}
\NormalTok{    n, m }\OperatorTok{=}\NormalTok{ mat.shape}
    \ControlFlowTok{if} \BuiltInTok{isinstance}\NormalTok{(k, }\BuiltInTok{tuple}\NormalTok{):}
\NormalTok{        k1, k2 }\OperatorTok{=}\NormalTok{ k}
    \ControlFlowTok{else}\NormalTok{:}
\NormalTok{        k1 }\OperatorTok{=}\NormalTok{ k2 }\OperatorTok{=}\NormalTok{ k}
    \ControlFlowTok{for}\NormalTok{ i }\KeywordTok{in} \BuiltInTok{range}\NormalTok{(k1, n}\OperatorTok{-}\NormalTok{k1):}
        \ControlFlowTok{for}\NormalTok{ j }\KeywordTok{in} \BuiltInTok{range}\NormalTok{(k, m}\OperatorTok{-}\NormalTok{k2):}
            \ControlFlowTok{if}\NormalTok{ method }\OperatorTok{==} \StringTok{'max'}\NormalTok{:}
\NormalTok{                condition }\OperatorTok{=}\NormalTok{ mat[i, j] }\OperatorTok{==}\NormalTok{ np.}\BuiltInTok{max}\NormalTok{(mat[i}\OperatorTok{-}\NormalTok{k1:i}\OperatorTok{+}\NormalTok{k1, j}\OperatorTok{-}\NormalTok{k2:j}\OperatorTok{+}\NormalTok{k2])}
            \ControlFlowTok{elif}\NormalTok{ method}\OperatorTok{==}\StringTok{'normal'}\NormalTok{:}
\NormalTok{                temp }\OperatorTok{=}\NormalTok{ mat[i}\OperatorTok{-}\NormalTok{k1:i}\OperatorTok{+}\NormalTok{k1, j}\OperatorTok{-}\NormalTok{k2:j}\OperatorTok{+}\NormalTok{k2]}
\NormalTok{                condition }\OperatorTok{=}\NormalTok{ mat[i, j] }\OperatorTok{>=}\NormalTok{ np.mean(temp) }\OperatorTok{+}\NormalTok{ t }\OperatorTok{*}\NormalTok{ np.std(temp)}
            \ControlFlowTok{elif}\NormalTok{ method }\OperatorTok{=} \StringTok{'max_linear'}\NormalTok{:}
\NormalTok{                condition }\OperatorTok{=}\NormalTok{ mat[i, j] }\OperatorTok{==}\NormalTok{ np.}\BuiltInTok{max}\NormalTok{(mat[i, j}\OperatorTok{-}\NormalTok{k1:j}\OperatorTok{+}\NormalTok{k2])}
            \ControlFlowTok{if}\NormalTok{ condition:}
\NormalTok{                mat2[i, j] }\OperatorTok{=} \DecValTok{1}
    \ControlFlowTok{return}\NormalTok{ mat2}
\end{Highlighting}
\end{Shaded}

\begin{enumerate}
\def\labelenumi{\arabic{enumi}.}
\setcounter{enumi}{1}
\tightlist
\item
  \textbf{Normal Thresholding}: in normal thresholding, if the pixel is
  larger that the average of the neighbors then it is set.
\end{enumerate}

    \begin{Verbatim}[commandchars=\\\{\}]
{\color{incolor}In [{\color{incolor}1}]:} \PY{k+kn}{import} \PY{n+nn}{numpy} \PY{k+kn}{as} \PY{n+nn}{np}
        \PY{k+kn}{import} \PY{n+nn}{cv2}
        \PY{k+kn}{from} \PY{n+nn}{library.utility} \PY{k+kn}{import} \PY{o}{*}
        \PY{k+kn}{import} \PY{n+nn}{matplotlib.pyplot} \PY{k+kn}{as} \PY{n+nn}{plt}
        \PY{k+kn}{from} \PY{n+nn}{iced} \PY{k+kn}{import} \PY{n}{normalization}
        \PY{k+kn}{from} \PY{n+nn}{iced} \PY{k+kn}{import} \PY{n+nb}{filter}
        \PY{k+kn}{from} \PY{n+nn}{sets} \PY{k+kn}{import} \PY{n}{Set}
        \PY{k+kn}{import} \PY{n+nn}{os}
        \PY{o}{\PYZpc{}}\PY{k}{load\PYZus{}ext} autoreload
        \PY{o}{\PYZpc{}}\PY{k}{autoreload} 2
        \PY{o}{\PYZpc{}}\PY{k}{pylab} inline
\end{Verbatim}


    \begin{Verbatim}[commandchars=\\\{\}]
Populating the interactive namespace from numpy and matplotlib

    \end{Verbatim}

    \begin{Verbatim}[commandchars=\\\{\}]
/home/bzr0014/git/watson/virt/local/lib/python2.7/site-packages/IPython/core/magics/pylab.py:161: UserWarning: pylab import has clobbered these variables: ['colors', 'log']
`\%matplotlib` prevents importing * from pylab and numpy
  "\textbackslash{}n`\%matplotlib` prevents importing * from pylab and numpy"

    \end{Verbatim}

    \section{Loading Data}\label{loading-data}

The first step is to load the raw Hi-C contact maps. We have already
extracted all inter- and intra-chromosomal contact maps and compiled
them into as single large numpy array format.

    \begin{Verbatim}[commandchars=\\\{\}]
{\color{incolor}In [{\color{incolor}2}]:} \PY{n}{pylab}\PY{o}{.}\PY{n}{rcParams}\PY{p}{[}\PY{l+s+s1}{\PYZsq{}}\PY{l+s+s1}{figure.figsize}\PY{l+s+s1}{\PYZsq{}}\PY{p}{]} \PY{o}{=} \PY{p}{(}\PY{l+m+mi}{15}\PY{p}{,} \PY{l+m+mi}{9}\PY{p}{)}
        
        \PY{n}{lengths\PYZus{}low\PYZus{}res} \PY{o}{=} \PY{n}{np}\PY{o}{.}\PY{n}{load}\PY{p}{(}\PY{l+s+s1}{\PYZsq{}}\PY{l+s+s1}{../numpy\PYZus{}data/length\PYZus{}low\PYZus{}res.npy}\PY{l+s+s1}{\PYZsq{}}\PY{p}{)}
        \PY{n}{mit\PYZus{}full}  \PY{o}{=} \PY{p}{(}\PY{n}{np}\PY{o}{.}\PY{n}{load}\PY{p}{(}\PY{l+s+s1}{\PYZsq{}}\PY{l+s+s1}{../numpy\PYZus{}data/mit\PYZus{}low\PYZus{}res.npy}\PY{l+s+s1}{\PYZsq{}}\PY{p}{)}\PY{p}{)}
        \PY{n}{all\PYZus{}full}  \PY{o}{=} \PY{p}{(}\PY{n}{np}\PY{o}{.}\PY{n}{load}\PY{p}{(}\PY{l+s+s1}{\PYZsq{}}\PY{l+s+s1}{../numpy\PYZus{}data/all\PYZus{}low\PYZus{}res.npy}\PY{l+s+s1}{\PYZsq{}}\PY{p}{)}\PY{p}{)}
        \PY{n}{rl\PYZus{}full}  \PY{o}{=} \PY{p}{(}\PY{n}{np}\PY{o}{.}\PY{n}{load}\PY{p}{(}\PY{l+s+s1}{\PYZsq{}}\PY{l+s+s1}{../numpy\PYZus{}data/rl\PYZus{}low\PYZus{}res.npy}\PY{l+s+s1}{\PYZsq{}}\PY{p}{)}\PY{p}{)}
        \PY{n}{call4\PYZus{}full}  \PY{o}{=} \PY{p}{(}\PY{n}{np}\PY{o}{.}\PY{n}{load}\PY{p}{(}\PY{l+s+s1}{\PYZsq{}}\PY{l+s+s1}{../numpy\PYZus{}data/call4\PYZus{}low\PYZus{}res.npy}\PY{l+s+s1}{\PYZsq{}}\PY{p}{)}\PY{p}{)}
        \PY{n}{data} \PY{o}{=} \PY{p}{\PYZob{}}\PY{p}{\PYZcb{}}
        \PY{n}{showImages}\PY{p}{(}\PY{p}{[}\PY{n}{mit\PYZus{}full} \PY{p}{,} \PY{n}{all\PYZus{}full} \PY{p}{,} \PY{n}{rl\PYZus{}full} \PY{p}{,} \PY{n}{call4\PYZus{}full} \PY{p}{]}\PY{p}{,} \PY{n}{rows}\PY{o}{=}\PY{l+m+mi}{2}\PY{p}{,} \PY{n}{titles}\PY{o}{=}\PY{p}{[}\PY{l+s+s1}{\PYZsq{}}\PY{l+s+s1}{MIT}\PY{l+s+s1}{\PYZsq{}}\PY{p}{,} \PY{l+s+s1}{\PYZsq{}}\PY{l+s+s1}{ALL}\PY{l+s+s1}{\PYZsq{}}\PY{p}{,} \PY{l+s+s1}{\PYZsq{}}\PY{l+s+s1}{RL}\PY{l+s+s1}{\PYZsq{}}\PY{p}{,} \PY{l+s+s1}{\PYZsq{}}\PY{l+s+s1}{CALL4}\PY{l+s+s1}{\PYZsq{}}\PY{p}{]}\PY{p}{)}
\end{Verbatim}


    \begin{Verbatim}[commandchars=\\\{\}]
Number of rows and columns: 2, 2
(4, 4)

    \end{Verbatim}

    \begin{center}
    \adjustimage{max size={0.9\linewidth}{0.9\paperheight}}{output_3_1.png}
    \end{center}
    { \hspace*{\fill} \\}
    
    \section{Selecting a inter-/intra-chromosomal contact
map}\label{selecting-a-inter-intra-chromosomal-contact-map}

You can chang chr1 and chr2 variables based on the contact map you want
to investigate.

    \begin{Verbatim}[commandchars=\\\{\}]
{\color{incolor}In [{\color{incolor}3}]:} \PY{n}{chr1} \PY{o}{=} \PY{l+m+mi}{1}
        \PY{n}{chr2} \PY{o}{=} \PY{l+m+mi}{2}
        \PY{n}{data}\PY{p}{[}\PY{l+s+s1}{\PYZsq{}}\PY{l+s+s1}{MIT}\PY{l+s+s1}{\PYZsq{}}\PY{p}{]}          \PY{o}{=} \PY{n}{get\PYZus{}contact\PYZus{}map}\PY{p}{(}\PY{n}{mit\PYZus{}full}        \PY{p}{,} \PY{n}{lengths\PYZus{}low\PYZus{}res}\PY{p}{,} \PY{n}{chr1}\PY{p}{,} \PY{n}{chr2}\PY{p}{)}
        \PY{n}{data}\PY{p}{[}\PY{l+s+s1}{\PYZsq{}}\PY{l+s+s1}{ALL}\PY{l+s+s1}{\PYZsq{}}\PY{p}{]}          \PY{o}{=} \PY{n}{get\PYZus{}contact\PYZus{}map}\PY{p}{(}\PY{n}{all\PYZus{}full}        \PY{p}{,} \PY{n}{lengths\PYZus{}low\PYZus{}res}\PY{p}{,} \PY{n}{chr1}\PY{p}{,} \PY{n}{chr2}\PY{p}{)}
        \PY{n}{data}\PY{p}{[}\PY{l+s+s1}{\PYZsq{}}\PY{l+s+s1}{RL}\PY{l+s+s1}{\PYZsq{}}\PY{p}{]}           \PY{o}{=} \PY{n}{get\PYZus{}contact\PYZus{}map}\PY{p}{(}\PY{n}{rl\PYZus{}full}         \PY{p}{,} \PY{n}{lengths\PYZus{}low\PYZus{}res}\PY{p}{,} \PY{n}{chr1}\PY{p}{,} \PY{n}{chr2}\PY{p}{)}
        \PY{n}{data}\PY{p}{[}\PY{l+s+s1}{\PYZsq{}}\PY{l+s+s1}{CALL4}\PY{l+s+s1}{\PYZsq{}}\PY{p}{]}        \PY{o}{=} \PY{n}{get\PYZus{}contact\PYZus{}map}\PY{p}{(}\PY{n}{call4\PYZus{}full}      \PY{p}{,} \PY{n}{lengths\PYZus{}low\PYZus{}res}\PY{p}{,} \PY{n}{chr1}\PY{p}{,} \PY{n}{chr2}\PY{p}{)}
        \PY{n}{showImages}\PY{p}{(}\PY{n}{data}\PY{p}{,} \PY{n}{rows} \PY{o}{=} \PY{l+m+mi}{2}\PY{p}{,} \PY{n}{titles}\PY{o}{=}\PY{p}{[}\PY{l+s+s1}{\PYZsq{}}\PY{l+s+s1}{MIT}\PY{l+s+s1}{\PYZsq{}}\PY{p}{,} \PY{l+s+s1}{\PYZsq{}}\PY{l+s+s1}{ALL}\PY{l+s+s1}{\PYZsq{}}\PY{p}{,} \PY{l+s+s1}{\PYZsq{}}\PY{l+s+s1}{RL}\PY{l+s+s1}{\PYZsq{}}\PY{p}{,} \PY{l+s+s1}{\PYZsq{}}\PY{l+s+s1}{CALL4}\PY{l+s+s1}{\PYZsq{}}\PY{p}{]}\PY{p}{)}
\end{Verbatim}


    \begin{Verbatim}[commandchars=\\\{\}]
Number of rows and columns: 2, 2
(4, 4)

    \end{Verbatim}

    \begin{center}
    \adjustimage{max size={0.9\linewidth}{0.9\paperheight}}{output_5_1.png}
    \end{center}
    { \hspace*{\fill} \\}
    
    \subsection{Cleaning Data}\label{cleaning-data}

As can be observed above, there are several rows and columns that simply
don't contain any data and are all zeros, in the following code we clean
the matrices to remove these zero rows and columns since they will cause
problems in future analysis.

    \begin{Verbatim}[commandchars=\\\{\}]
{\color{incolor}In [{\color{incolor}4}]:} \PY{c+c1}{\PYZsh{}There are some blank rows and columns in the matrix. let\PYZsq{}s remove them.}
        \PY{n}{blankRows0}\PY{p}{,} \PY{n}{blankCols0} \PY{o}{=} \PY{n}{getBlankRowsAndColumns}\PY{p}{(}\PY{n}{data}\PY{p}{[}\PY{l+s+s1}{\PYZsq{}}\PY{l+s+s1}{MIT}\PY{l+s+s1}{\PYZsq{}}\PY{p}{]}\PY{p}{)}
        \PY{n}{blankRows1}\PY{p}{,} \PY{n}{blankCols1} \PY{o}{=} \PY{n}{getBlankRowsAndColumns}\PY{p}{(}\PY{n}{data}\PY{p}{[}\PY{l+s+s1}{\PYZsq{}}\PY{l+s+s1}{ALL}\PY{l+s+s1}{\PYZsq{}}\PY{p}{]}\PY{p}{)}
        \PY{n}{blankRows2}\PY{p}{,} \PY{n}{blankCols2} \PY{o}{=} \PY{n}{getBlankRowsAndColumns}\PY{p}{(}\PY{n}{data}\PY{p}{[}\PY{l+s+s1}{\PYZsq{}}\PY{l+s+s1}{RL}\PY{l+s+s1}{\PYZsq{}}\PY{p}{]}\PY{p}{)}
        \PY{n}{blankRows3}\PY{p}{,} \PY{n}{blankCols3} \PY{o}{=} \PY{n}{getBlankRowsAndColumns}\PY{p}{(}\PY{n}{data}\PY{p}{[}\PY{l+s+s1}{\PYZsq{}}\PY{l+s+s1}{CALL4}\PY{l+s+s1}{\PYZsq{}}\PY{p}{]}\PY{p}{)}
        \PY{n}{blankRows} \PY{o}{=} \PY{n}{Set}\PY{p}{(}\PY{p}{[}\PY{p}{]}\PY{p}{)}
        \PY{n}{blankCols} \PY{o}{=} \PY{n}{Set}\PY{p}{(}\PY{p}{[}\PY{p}{]}\PY{p}{)}
        \PY{n}{blankRows}\PY{o}{.}\PY{n}{update}\PY{p}{(}\PY{n}{blankRows0}\PY{p}{)}
        \PY{n}{blankRows}\PY{o}{.}\PY{n}{update}\PY{p}{(}\PY{n}{blankRows1}\PY{p}{)}
        \PY{n}{blankRows}\PY{o}{.}\PY{n}{update}\PY{p}{(}\PY{n}{blankRows2}\PY{p}{)}
        \PY{n}{blankRows}\PY{o}{.}\PY{n}{update}\PY{p}{(}\PY{n}{blankRows3}\PY{p}{)}
        \PY{n}{blankCols}\PY{o}{.}\PY{n}{update}\PY{p}{(}\PY{n}{blankCols0}\PY{p}{)}
        \PY{n}{blankCols}\PY{o}{.}\PY{n}{update}\PY{p}{(}\PY{n}{blankCols1}\PY{p}{)}
        \PY{n}{blankCols}\PY{o}{.}\PY{n}{update}\PY{p}{(}\PY{n}{blankCols2}\PY{p}{)}
        \PY{n}{blankCols}\PY{o}{.}\PY{n}{update}\PY{p}{(}\PY{n}{blankCols3}\PY{p}{)}
        \PY{n}{data}\PY{p}{[}\PY{l+s+s1}{\PYZsq{}}\PY{l+s+s1}{MIT}\PY{l+s+s1}{\PYZsq{}}\PY{p}{]} \PY{o}{=} \PY{n}{removeRowsAndColumns}\PY{p}{(}\PY{n}{data}\PY{p}{[}\PY{l+s+s1}{\PYZsq{}}\PY{l+s+s1}{MIT}\PY{l+s+s1}{\PYZsq{}}\PY{p}{]}\PY{p}{,} \PY{n}{blankRows}\PY{p}{,} \PY{n}{blankCols}\PY{p}{)}
        \PY{n}{data}\PY{p}{[}\PY{l+s+s1}{\PYZsq{}}\PY{l+s+s1}{ALL}\PY{l+s+s1}{\PYZsq{}}\PY{p}{]} \PY{o}{=} \PY{n}{removeRowsAndColumns}\PY{p}{(}\PY{n}{data}\PY{p}{[}\PY{l+s+s1}{\PYZsq{}}\PY{l+s+s1}{ALL}\PY{l+s+s1}{\PYZsq{}}\PY{p}{]}\PY{p}{,} \PY{n}{blankRows}\PY{p}{,} \PY{n}{blankCols}\PY{p}{)}
        \PY{n}{data}\PY{p}{[}\PY{l+s+s1}{\PYZsq{}}\PY{l+s+s1}{RL}\PY{l+s+s1}{\PYZsq{}}\PY{p}{]} \PY{o}{=} \PY{n}{removeRowsAndColumns}\PY{p}{(}\PY{n}{data}\PY{p}{[}\PY{l+s+s1}{\PYZsq{}}\PY{l+s+s1}{RL}\PY{l+s+s1}{\PYZsq{}}\PY{p}{]}\PY{p}{,} \PY{n}{blankRows}\PY{p}{,} \PY{n}{blankCols}\PY{p}{)}
        \PY{n}{data}\PY{p}{[}\PY{l+s+s1}{\PYZsq{}}\PY{l+s+s1}{CALL4}\PY{l+s+s1}{\PYZsq{}}\PY{p}{]} \PY{o}{=} \PY{n}{removeRowsAndColumns}\PY{p}{(}\PY{n}{data}\PY{p}{[}\PY{l+s+s1}{\PYZsq{}}\PY{l+s+s1}{CALL4}\PY{l+s+s1}{\PYZsq{}}\PY{p}{]}\PY{p}{,} \PY{n}{blankRows}\PY{p}{,} \PY{n}{blankCols}\PY{p}{)}
        \PY{n}{n1}\PY{p}{,} \PY{n}{m1} \PY{o}{=} \PY{n}{data}\PY{p}{[}\PY{l+s+s1}{\PYZsq{}}\PY{l+s+s1}{MIT}\PY{l+s+s1}{\PYZsq{}}\PY{p}{]}\PY{o}{.}\PY{n}{shape}
        \PY{n}{n2}\PY{p}{,} \PY{n}{m2} \PY{o}{=} \PY{n}{data}\PY{p}{[}\PY{l+s+s1}{\PYZsq{}}\PY{l+s+s1}{RL}\PY{l+s+s1}{\PYZsq{}}\PY{p}{]}\PY{o}{.}\PY{n}{shape}
        \PY{n}{n3}\PY{p}{,} \PY{n}{m3} \PY{o}{=} \PY{n}{data}\PY{p}{[}\PY{l+s+s1}{\PYZsq{}}\PY{l+s+s1}{ALL}\PY{l+s+s1}{\PYZsq{}}\PY{p}{]}\PY{o}{.}\PY{n}{shape}
        \PY{n}{n4}\PY{p}{,} \PY{n}{m4} \PY{o}{=} \PY{n}{data}\PY{p}{[}\PY{l+s+s1}{\PYZsq{}}\PY{l+s+s1}{CALL4}\PY{l+s+s1}{\PYZsq{}}\PY{p}{]}\PY{o}{.}\PY{n}{shape}
\end{Verbatim}


    \begin{Verbatim}[commandchars=\\\{\}]
('size of old matrix:', (495, 486))
('size of new matrix:', (454, 479))
('size of old matrix:', (495, 486))
('size of new matrix:', (454, 479))
('size of old matrix:', (495, 486))
('size of new matrix:', (454, 479))
('size of old matrix:', (495, 486))
('size of new matrix:', (454, 479))

    \end{Verbatim}

    \subsubsection{Comparing CALL4 with RL low-resolution
data:}\label{comparing-call4-with-rl-low-resolution-data}

Feel free to change D1 and D2 to pick any data from set of ```{[}'MIT',
'ALL', 'RL', 'CALL4'{]}````.

    \begin{Verbatim}[commandchars=\\\{\}]
{\color{incolor}In [{\color{incolor}6}]:} \PY{n}{D1} \PY{o}{=} \PY{l+s+s1}{\PYZsq{}}\PY{l+s+s1}{CALL4}\PY{l+s+s1}{\PYZsq{}}
        \PY{n}{D2} \PY{o}{=} \PY{l+s+s1}{\PYZsq{}}\PY{l+s+s1}{RL}\PY{l+s+s1}{\PYZsq{}}
        \PY{n}{pylab}\PY{o}{.}\PY{n}{rcParams}\PY{p}{[}\PY{l+s+s1}{\PYZsq{}}\PY{l+s+s1}{figure.figsize}\PY{l+s+s1}{\PYZsq{}}\PY{p}{]} \PY{o}{=} \PY{p}{(}\PY{l+m+mi}{15}\PY{p}{,} \PY{l+m+mi}{20}\PY{p}{)}
        \PY{c+c1}{\PYZsh{} Size of the kernel}
        \PY{n}{k} \PY{o}{=} \PY{p}{(}\PY{l+m+mi}{2}\PY{p}{,} \PY{l+m+mi}{2}\PY{p}{,} \PY{l+m+mi}{2}\PY{p}{,} \PY{l+m+mi}{2}\PY{p}{)}
        \PY{c+c1}{\PYZsh{} can be either \PYZsq{}max\PYZsq{} for setting the maximum value in each kernel to 1 and the rest to 0}
        \PY{c+c1}{\PYZsh{} or \PYZsq{}normal\PYZsq{} for setting all values above mean + t * std withing the kernel to 1 and }
        \PY{c+c1}{\PYZsh{} the rest to 0}
        \PY{n}{method} \PY{o}{=} \PY{l+s+s1}{\PYZsq{}}\PY{l+s+s1}{normal}\PY{l+s+s1}{\PYZsq{}}
        \PY{c+c1}{\PYZsh{} in case of normal thresholding, t is the coefficient }
        \PY{c+c1}{\PYZsh{} of the standard deviation; that is, n each kernel iteration K}
        \PY{c+c1}{\PYZsh{} if K[i, j] \PYZgt{} mean(K) + t * std(K), then it is set to 1.}
        \PY{n}{t} \PY{o}{=} \PY{l+m+mi}{0}
        \PY{n}{params} \PY{o}{=} \PY{n+nb+bp}{None}
        \PY{n}{symmetric}\PY{o}{=} \PY{n}{chr1}\PY{o}{==}\PY{n}{chr2}
        \PY{n}{D1\PYZus{}os} \PY{o}{=} \PY{n}{local\PYZus{}threshold}\PY{p}{(}\PY{p}{(}\PY{p}{(}\PY{n}{data}\PY{p}{[}\PY{n}{D1}\PY{p}{]}\PY{o}{+}\PY{l+m+mi}{1}\PY{o}{+}\PY{l+m+mf}{1e\PYZhy{}5}\PY{p}{)}\PY{p}{)}\PY{p}{,} \PY{n}{k} \PY{o}{=} \PY{n}{k}\PY{p}{,} \PY{n}{method}\PY{o}{=}\PY{n}{method}\PY{p}{,} \PY{n}{t} \PY{o}{=} \PY{n}{t}\PY{p}{,} \PY{n}{params}\PY{o}{=}\PY{n}{params}\PY{p}{,} \PY{n}{symmetric}\PY{o}{=}\PY{n}{symmetric}\PY{p}{)}
        \PY{n}{D2\PYZus{}os} \PY{o}{=} \PY{n}{local\PYZus{}threshold}\PY{p}{(}\PY{p}{(}\PY{p}{(}\PY{n}{data}\PY{p}{[}\PY{n}{D2}\PY{p}{]}\PY{o}{+}\PY{l+m+mi}{1}\PY{o}{+}\PY{l+m+mf}{1e\PYZhy{}5}\PY{p}{)}\PY{p}{)}\PY{p}{,} \PY{n}{k} \PY{o}{=} \PY{n}{k}\PY{p}{,} \PY{n}{method}\PY{o}{=}\PY{n}{method}\PY{p}{,} \PY{n}{t} \PY{o}{=} \PY{n}{t}\PY{p}{,} \PY{n}{params}\PY{o}{=}\PY{n}{params}\PY{p}{,} \PY{n}{symmetric}\PY{o}{=}\PY{n}{symmetric}\PY{p}{)}
        \PY{n}{n1}\PY{p}{,} \PY{n}{m1} \PY{o}{=} \PY{n}{D1\PYZus{}os}\PY{o}{.}\PY{n}{shape}
        \PY{n}{n2}\PY{p}{,} \PY{n}{m2} \PY{o}{=} \PY{n}{D2\PYZus{}os}\PY{o}{.}\PY{n}{shape}
        \PY{n}{n} \PY{o}{=} \PY{n}{np}\PY{o}{.}\PY{n}{min}\PY{p}{(}\PY{p}{[}\PY{n}{n1}\PY{p}{,} \PY{n}{n2}\PY{p}{]}\PY{p}{)}
        \PY{n}{m} \PY{o}{=} \PY{n}{np}\PY{o}{.}\PY{n}{min}\PY{p}{(}\PY{p}{[}\PY{n}{m1}\PY{p}{,} \PY{n}{m2}\PY{p}{]}\PY{p}{)}
        \PY{n}{D1\PYZus{}os} \PY{o}{=} \PY{n}{D1\PYZus{}os}\PY{p}{[}\PY{p}{:}\PY{n}{n}\PY{p}{,} \PY{p}{:}\PY{n}{m}\PY{p}{]}
        \PY{n}{D1\PYZus{}data} \PY{o}{=} \PY{n}{data}\PY{p}{[}\PY{n}{D1}\PY{p}{]}\PY{p}{[}\PY{p}{:}\PY{n}{n}\PY{p}{,} \PY{p}{:}\PY{n}{m}\PY{p}{]}
        \PY{n}{D2\PYZus{}os} \PY{o}{=} \PY{n}{D2\PYZus{}os}\PY{p}{[}\PY{p}{:}\PY{n}{n}\PY{p}{,} \PY{p}{:}\PY{n}{m}\PY{p}{]}
        \PY{n}{D2\PYZus{}data} \PY{o}{=} \PY{n}{data}\PY{p}{[}\PY{n}{D2}\PY{p}{]}\PY{p}{[}\PY{p}{:}\PY{n}{n}\PY{p}{,} \PY{p}{:}\PY{n}{m}\PY{p}{]}
        \PY{n}{a} \PY{o}{=} \PY{p}{(}\PY{p}{(}\PY{n}{D1\PYZus{}os} \PY{o}{*} \PY{n}{D1\PYZus{}data}\PY{p}{)} \PY{o}{\PYZgt{}} \PY{l+m+mi}{0}\PY{p}{)} \PY{o}{*} \PY{l+m+mi}{1}
        \PY{n}{b} \PY{o}{=} \PY{p}{(}\PY{p}{(}\PY{n}{D2\PYZus{}os} \PY{o}{*} \PY{n}{D2\PYZus{}data}\PY{p}{)} \PY{o}{\PYZgt{}} \PY{l+m+mi}{0}\PY{p}{)} \PY{o}{*} \PY{l+m+mi}{1}
\end{Verbatim}


    \begin{Verbatim}[commandchars=\\\{\}]
(2, 2, 2, 2)
(2, 2, 2, 2)

    \end{Verbatim}

    \begin{Verbatim}[commandchars=\\\{\}]
{\color{incolor}In [{\color{incolor}8}]:} \PY{n}{images} \PY{o}{=} \PY{p}{[} \PY{n}{D1\PYZus{}os}\PY{p}{,} \PY{n}{a}\PY{p}{,} \PY{p}{(}\PY{n}{a}\PY{o}{\PYZhy{}}\PY{n}{b}\PY{p}{)} \PY{o}{\PYZgt{}} \PY{l+m+mi}{0}\PY{p}{,} \PY{n}{D2\PYZus{}os}\PY{p}{,}  \PY{n}{b}\PY{p}{,} \PY{n}{a} \PY{o}{*} \PY{n}{b} \PY{p}{]}
        \PY{n}{titles} \PY{o}{=} \PY{p}{[}\PY{n}{D1}\PY{p}{,} \PY{l+s+s1}{\PYZsq{}}\PY{l+s+s1}{Thresholded }\PY{l+s+si}{\PYZpc{}s}\PY{l+s+s1}{\PYZsq{}}\PY{o}{\PYZpc{}}\PY{k}{D1}, \PYZsq{}Difference\PYZsq{}, D2, \PYZsq{}Thresholded \PYZpc{}s\PYZsq{}\PYZpc{}D2, \PYZsq{}Similarity\PYZsq{}]
        \PY{n}{showImages}\PY{p}{(}\PY{n}{images}\PY{p}{,} \PY{l+m+mi}{2}\PY{p}{,} \PY{n}{titles}\PY{o}{=}\PY{n}{titles}\PY{p}{)}
\end{Verbatim}


    \begin{Verbatim}[commandchars=\\\{\}]
Number of rows and columns: 2, 3
(6, 6)

    \end{Verbatim}

    \begin{center}
    \adjustimage{max size={0.9\linewidth}{0.9\paperheight}}{output_10_1.png}
    \end{center}
    { \hspace*{\fill} \\}
    

    % Add a bibliography block to the postdoc
    
    
    
    \end{document}
