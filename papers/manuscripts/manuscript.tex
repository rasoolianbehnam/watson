\documentclass[a4,center,fleqn]{NAR}
\usepackage{hyperref}
\usepackage{graphicx}
\usepackage{csquotes}
\usepackage{amsmath}
\usepackage{setspace}
\usepackage{float}
\usepackage{caption}

\begin{document}
\section{Abstract}

In this study, we plan to find dissimilarities between normal cells and cancerous cells,
through investigating HiC contact maps. 
We suspect that there are systematic differences between how chromosomes are structured
between normal cells and cancerous cells.
we 
Ideally, it is desirable to compare 3D structures of 
cell in order to make such comparisons.
However, the main challenge that we face is that 
3D structure of a cell is not readily available. Based on
\cite{adhikari2016chromosome3d}, fluorescence in situ hybridizaiton
(FISH) is used for investigating 3D configuration of chromosomes.
However, this method can only be used locally and cannot map
the whole structure of the chromosomes.
In orther to find dissimilarities in the 3D structure of 
chromosomes, we used HiC dataset.
The HiC method, which was developed by \cite{lieberman2009comprehensive}, captures interactions between 
chromosomal fragments in kilobase resolution. Based on HiC data, an
\textit{interaction frequency (IF) } matrix can be developed between \textit{loci} at a desired resolution.
A cell $IF_{ij}$ in an interaction frequency matrix captures the number of interaction detected
in HiC dataset between locus $i$ and locus $j$ in the genome.
An interaction matrix can be used to develop both inter- and intra-chromosomal interaction matrices.
We believe differences in interaction matrices can be found between normal cells and cancerous ones.

\section{Introduction}

Graphlet comparison is a novel method used to compare large networks in order to
find local similarities in them.
Authors of \cite{prvzulj2007biological} provide a new measure of PPI
network comparison
based on 73 constraints. This is used in order to compare two large
networks in order to detect similarities.

\cite{milenkoviae2008uncovering} 
 provide heuristics to compare two nodes based on some feature
(or signature) vectors, which is a 73-dimensional vector
$\mathbf{s}^T
= [s_0, s_2, ..., s_{72}]$ where $s_i$ denotes the number of nodes in
the network that are part of an orbit $i$. \\
\textit{Important Result}: Proteins with similar surroundings perform
similar functions.

In \cite{milenkovic2010cancer}, the same author investigates 
cancer-causing genes to find similarities in their signatures. After
clustering the genes based on \textit{signature similarity} criteria,
some clusters contain a lot of cancerous genes.
They use 4 different clustering methods with varying parameters to cluster
the proteins. They then predict the cancer-relatedness of a protein 
$i$ using
an enrichment criteria $\frac{k}{|C_i|}$ where $C_i$ is the cluster
where protein $i$ belongs and $k$ is the number of cancer-causing
proteins in $C_i$ and $|C_i|$ is the size of $C_i$.

Implementations of algorithms of extracting graphlets: \\
\begin{itemize}
    \item GraphCrunch: 
        \url{http://www0.cs.ucl.ac.uk/staff/natasa/graphcrunch2/usage.html}
    \item PGD: \url{http://nesreenahmed.com/graphlets/}
    \item ORCA: Graphlet and orbit counting algorithm \\
        \url{ https://CRAN.R-project.org/package=orca} \\
        This package is in R. In order to install it, type
        \texttt{install.packages("orca")}.
        
\end{itemize}

The authors of \cite{di2010fast} generalized the idea of graphlets to 
ordered graphs were the nodes are labeled in ascending order.
These graphlets are illustrated in Figure \ref{fig:ordered_graphlets}.
As can be viewed, there are a total of 14 orbits for graphlets of size
2 and 3 since the label of graphlets is also included in toplogy.
In the new definition, $d_v^i$ denotes the number of orbit $i$ touches 
node $v$. Each node, is then assigned a vector of length 14 
\footnote{number of orbits in graphlets of size 2 and 3}
$(d_v^1, d_v^2, ..., d_v^{14})$ 
and similarity of two nodes in two contact maps can be compared by
how geometrically close their corresponding vectors are.
\section{Materials and Methods}
\subsection{Thresholding contact maps}
In order to be able to extract graphlets, HiC contact maps should be modeled as
unweighted graphs where the nodes represent the loci and an edge between two 
nodes represent a \textit{significant} interaction between the loci the nodes
represent.

Thresholding is achieved by thresholding the contact maps. The result
of the thresholding procedure would be a binary matrix which also can serve as
an adjacency matrix for an unweighted, undirected graph. The graph can then be
used for orbit extraction.

In order to go about the process of thresholding, it is necessary to make sure
that both global and local features are maintained. We could consider 
thresholding the contact maps by simply setting values above a fixed value to
one and the rest to zero. However, in practice, this proved result in graphs
that capture the local structure of the contact maps poorly. This is because
intensities follow an exponential distribution with a mean close to zero and
some very larges values that correspond t interactions along and close to 
the main diagonal of the contact maps.
Thus, picking relatively large numbers would result in ignoring interactions
that are far from the main diagonal and picking small numbers will lead to
capturing too many \textit{insignificant} interactions.

In order to threshold the matrix so that both global and local patterns are
kept, we borrowed the concept of \textit{adaptive thresholding} from image 
processing context. In this method, in order to be set, a pixel should have
an intensity that is larger than the average of non-zero intensities in its
\textit{neighborhood}. The neighborhood is defined by an sliding kernel 
that passes through the contact map with a pixel at its middle at 
each step.




\section{Resources}
\textbf{Publications related to Hi-C:}
\begin{enumerate}
    \item \url{https://www.ncbi.nlm.nih.gov/pmc/articles/PMC2858594/}
    \item \url{http://journals.plos.org/plosone/article?id=10.1371/journal.pone.0058793 }
    \item \url{http://nar.oxfordjournals.org/content/42/7/e52.full}
    \item \url{http://bioinformatics.oxfordjournals.org/content/early/2015/12/31/bioinformatics.btv754.abstract?keytype=ref&ijkey=A97WhKqBiEIcuzd}
    \item \url{https://www.ncbi.nlm.nih.gov/pmc/articles/PMC4417147/}
    \item \url{http://www.pnas.org/content/112/47/E6456.full}
    \item \url{http://www.pnas.org/content/113/12/E1663.full}
\end{enumerate}
\textbf{Hi-C Datasets:}
\begin{enumerate}
    \item Original Datasets: \url{https://bcm.app.box.com/v/aidenlab/folder/11234760671}
    \item Including cancerous cells: \url{http://sysbio.rnet.missouri.edu/T0510/tmp_download/link_to_download_genome_data/}
    \item Chromosome3D project: \url{http://sysbio.rnet.missouri.edu/bdm_download/chromosome3d/}
\end{enumerate}
\textbf{Contact Matrix Analysis:}
\begin{enumerate}
    \item \url{https://omictools.com/contact-matrix-normalization-category}
    \item \url{http://hifive.docs.taylorlab.org/en/latest/}
\end{enumerate}

\textbf{Labs working on 3D Human Genome:}

\begin{enumerate}
    \item \url{http://mirnylab.mit.edu}
    \item \url{http://dostielab.biochem.mcgill.ca}
    \item \url{http://www.aidenlab.org/}
    \item \url{http://web.cmb.usc.edu/people/alber/index.htm}
    \item \url{http://calla.rnet.missouri.edu/cheng/nsf_career.html}
\end{enumerate}
\textbf{Resources related to Graphlet:}

\begin{enumerate}
    \item \url{https://en.wikipedia.org/wiki/Graphlets}
    \item \url{https://academic.oup.com/bioinformatics/article/23/2/e177/202080/Biological-network-comparison-using-graphlet}
    \item \url{http://www0.cs.ucl.ac.uk/staff/N.Przulj/index.html}
\end{enumerate}

\bibliography{lit}
\bibliographystyle{unsrt}
\end{document}
